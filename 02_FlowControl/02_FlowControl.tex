%=============================================================================%
% Author: 	John Joseph Valletta
% Date: 	09/04/2017
% Title: 	Python workshop: Flow control
%=============================================================================%

%=============================================================================%
% Preamble
%=============================================================================%
% Libraries
\documentclass[pdf]{beamer}
\usepackage[export]{adjustbox}
\usepackage{framed}
\usepackage{color}
\definecolor{dkgreen}{rgb}{0,0.6,0}
\definecolor{gray}{rgb}{0.5,0.5,0.5}
\definecolor{mauve}{rgb}{0.58,0,0.82}
\definecolor{deepblue}{rgb}{0,0,0.5}
\definecolor{deepred}{rgb}{0.6,0,0}
\definecolor{deepgreen}{rgb}{0,0.5,0}
\definecolor{lightgray}{rgb}{0.92,0.92,0.92}
\usepackage{listings} % to insert code
\usepackage{textpos} % textblock
\usepackage{hyperref}
\hypersetup{colorlinks=true, urlcolor=blue, linkcolor=black} 

% Listing set up
% Python
\lstdefinestyle{python}{
language=python,
formfeed=\newpage,
basicstyle=\scriptsize\ttfamily,
commentstyle=\color{deepgreen},%\color{gray},
%numbers=left,
%numberstyle=\tiny\color{gray},
stepnumber=1,
numbersep=5pt,
backgroundcolor=\color{lightgray},%\color{white},
showspaces=false,
showstringspaces=false,
showtabs=false,
frame=lines,
tabsize=4,
captionpos=b,
breaklines=true,
breakatwhitespace=false,
title=\lstname,
escapeinside={},
keywordstyle=\color{deepblue},
emphstyle=\color{deepred},
stringstyle=\color{mauve},
morekeywords={as, lambda, in, True, False}
}

% Presentation configuration
\mode<presentation>{\usetheme{Madrid}}
\definecolor{tealblue}{rgb}{0, 0.5, 0.5}
\usecolortheme[named=tealblue]{structure}
\useinnertheme{circles} % circles, rectanges, rounded, inmargin
\usefonttheme[onlymath]{serif} % makes math fonts like the usual LaTeX ones
\setbeamercovered{transparent=4} % transparent
\setbeamertemplate{caption}{\raggedright\insertcaption\par} % Remove the word "Figure" from caption %\setbeamertemplate{caption}[default]
\setbeamertemplate{navigation symbols}{} % don't put navigation tools at the bottom (alternatively \beamertemplatenavigationsymbolsempty)
\graphicspath{ {../images/} }

% Titlepage
\title[Python for scientific research]{Python for scientific research}
\subtitle{Flow control}
\author{John Joseph Valletta}
\date[June 2017]{June 2017}
\institute[]{University of Exeter, Penryn Campus, UK}
\titlegraphic{
\hfill
\includegraphics[width=\textwidth, keepaspectratio]{logo.jpg}}

%=============================================================================%
%=============================================================================%
% Start of Document
%=============================================================================%
%=============================================================================%
\begin{document}

%=============================================================================%
%=============================================================================%
\begin{frame}
\titlepage
\end{frame}

%=============================================================================%
%=============================================================================%
\begin{frame}{What we've done so far}

	\begin{enumerate}\addtolength{\itemsep}{1\baselineskip}
		\item Declare variables using built-in data types and execute operations
		on them
		\item \textbf{Next}: Controlling the flow of a program
	\end{enumerate}

\end{frame}

%=============================================================================%
%=============================================================================%
\begin{frame}{Motivation}

\begin{itemize}\addtolength{\itemsep}{2\baselineskip}
	\item<1-> Executing code one line at a time is useful but limiting
	\item<2-> \textbf{Flow control} commands lets us dictate the \textbf{order}
	in which commands are run: 
	\begin{enumerate}\addtolength{\itemsep}{1\baselineskip}
		\item<3-> \textbf{If-else}: to change what commands are executed
		under certain conditions
		\item<4-> \textbf{For loops}: to repeat the same thing $N$ times
		\item<5-> \textbf{While loops}: to repeat the same thing until a specific condition is met
	\end{enumerate}
\end{itemize}

\end{frame}

%=============================================================================%
%=============================================================================%
\begin{frame}[fragile]
\frametitle{If-else}

\begin{itemize}
	\item Print whether the integer $x$ is positive, negative or zero

\begin{lstlisting}[style=python]
if x > 0:
    print("x is positive")
elif x < 0:
    print("x is negative")
else:
    print("x is zero")
\end{lstlisting}

	\item Note the lack of \{ \} used in other languages; in Python \textbf{indentation} is everything!
\end{itemize}

\end{frame}

%=============================================================================%
%=============================================================================%
\begin{frame}[fragile]
\frametitle{For loops}

\begin{itemize}

\item<1-> Print the integers 1 to 5
\begin{lstlisting}[style=python]
for x in range(5):
    print(x+1)
\end{lstlisting}

\item<2-> Loop through a list of gene names and print them in upper case
\begin{lstlisting}[style=python]
geneNames = ["Irf1", "Ccl3", "Il12rb1", "Ifng", "Cxcl10"]
for gene in geneNames:
    print(gene.upper())
\end{lstlisting}

\end{itemize}

\end{frame}

%=============================================================================%
%=============================================================================%
\begin{frame}[fragile]
\frametitle{While loops}

\begin{itemize}

\item Print the integers 10 to 1
\begin{lstlisting}[style=python]
x = 10
while x > 0:
    print(x)
    x = x - 1
\end{lstlisting}

\item \textbf{Note}: 
\begin{enumerate}\addtolength{\itemsep}{0.5\baselineskip}
\item Use \texttt{for loops} over \texttt{while loops} where possible
\item Ensure that the \texttt{while} condition evaluates to 
\texttt{False} at some point to avoid an infinite loop
\end{enumerate}

\end{itemize}

\end{frame}

%=============================================================================%
%=============================================================================%
\begin{frame}[fragile]
\frametitle{List Comprehensions}

\begin{itemize}\addtolength{\itemsep}{0.5\baselineskip}

\item<1-> List comprehensions are an optimized and readible method for creating a list

\item<2-> Recall the previous example where we looped over gene names and printed them in upper
case
\begin{lstlisting}[style=python]
geneNames = ["Irf1", "Ccl3", "Il12rb1", "Ifng", "Cxcl10"]
for gene in geneNames:
    print(gene.upper())
\end{lstlisting}

\item<3-> What if I want to store the upper case gene names in another variable, 
called \texttt{x} for simplicity?

\end{itemize}

\end{frame}

%=============================================================================%
%=============================================================================%
\begin{frame}[fragile]
\frametitle{List Comprehensions}

\begin{enumerate}
\item<1-> Using for loops:
\begin{lstlisting}[style=python]
x = [] # create an empty list to append to
for gene in geneNames:
    x.append(gene.upper())
\end{lstlisting}

\item<2-> Using list comprehension:
\begin{lstlisting}[style=python]
x = [gene.upper() for gene in geneNames]
\end{lstlisting}

\item<3-> What if I want to ignore gene \texttt{Ifng}?
\begin{lstlisting}[style=python]
x = [gene.upper() for gene in geneNames if gene != "Ifng"]
\end{lstlisting}

\end{enumerate}

\end{frame}

%=============================================================================%
%=============================================================================%
\begin{frame}[fragile]
\frametitle{Enumerate}

\begin{itemize}\addtolength{\itemsep}{0.5\baselineskip}
\item<1-> When looping over lists, sometimes it's useful to keep track of the
index of the iteration

\item<2-> \textbf{Enumerate} is a built-in function that lets us access the
iterable element but also its index

\item<3-> Print the index next to upper cased gene name

\begin{enumerate}\addtolength{\itemsep}{-1\baselineskip}

\item<3-> Using a standard for loop:
\begin{lstlisting}[style=python]
i = 0 # index counter
for gene in geneNames:
    print("{0}. {1}\n".format(i+1, gene.upper()))
    i = i + 1
\end{lstlisting}

\item<4-> Using \texttt{enumerate}:
\begin{lstlisting}[style=python]
for i, gene in enumerate(geneNames):
    print("{0}. {1}\n".format(i+1, gene.upper()))
\end{lstlisting}

\end{enumerate}
\end{itemize}

\end{frame}

%=============================================================================%
%=============================================================================%
% End of Document
%=============================================================================%
%=============================================================================%
\end{document}

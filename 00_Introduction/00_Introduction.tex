%=============================================================================%
% Author: 	John Joseph Valletta
% Date: 	10/03/2017
% Title: 	Introduction to Python Workshop
%=============================================================================%

%=============================================================================%
% Preamble
%=============================================================================%
% Libraries
\documentclass[pdf]{beamer}
\usepackage[export]{adjustbox}
\usepackage{framed}
\usepackage{color}
\definecolor{dkgreen}{rgb}{0,0.6,0}
\definecolor{gray}{rgb}{0.5,0.5,0.5}
\definecolor{mauve}{rgb}{0.58,0,0.82}
\definecolor{deepblue}{rgb}{0,0,0.5}
\definecolor{deepred}{rgb}{0.6,0,0}
\definecolor{deepgreen}{rgb}{0,0.5,0}
\definecolor{lightgray}{rgb}{0.92,0.92,0.92}
\usepackage{listings} % to insert code
\usepackage{textpos} % textblock
\usepackage{hyperref}
\hypersetup{colorlinks=true, urlcolor=blue, linkcolor=black} 

% Listing set up
% bash
\lstdefinestyle{bash}{
language=bash,                     % the language of the code
basicstyle=\scriptsize\ttfamily,       % the size of the fonts that are used for the code
numbers=none,%left,                   % where to put the line-numbers
numberstyle=\tiny\color{gray},  % the style that is used for the line-numbers
stepnumber=1,                   % the step between two line-numbers. If it's 1, each line
                          % will be numbered
numbersep=5pt,                  % how far the line-numbers are from the code
backgroundcolor=\color{lightgray},  % choose the background color. You must add \usepackage{color}
showspaces=false,               % show spaces adding particular underscores
showstringspaces=false,         % underline spaces within strings
showtabs=false,                 % show tabs within strings adding particular underscores
frame=lines,%single,                   % adds a frame around the code
rulecolor=\color{black},        % if not set, the frame-color may be changed on line-breaks within not-black text (e.g. commens (green here))
tabsize=2,                      % sets default tabsize to 2 spaces
captionpos=b,                   % sets the caption-position to bottom
breaklines=true,                % sets automatic line breaking
breakatwhitespace=false,        % sets if automatic breaks should only happen at whitespace
title=\lstname,                 % show the filename of files included with \lstinputlisting;
                          % also try caption instead of title
keywordstyle=\color{blue},      % keyword style
commentstyle=\color{dkgreen},   % comment style
stringstyle=\color{mauve},      % string literal style
escapeinside={\%*}{*)},         % if you want to add a comment within your code
morekeywords={}            % if you want to add more keywords to the set
}

% Python
\lstdefinestyle{python}{
language=python,
formfeed=\newpage,
basicstyle=\scriptsize\ttfamily,
commentstyle=\color{deepgreen},%\color{gray},
numbers=left,
numberstyle=\tiny\color{gray},
stepnumber=1,
numbersep=5pt,
backgroundcolor=\color{lightgray},%\color{white},
showspaces=false,
showstringspaces=false,
showtabs=false,
frame=lines,
tabsize=4,
captionpos=b,
breaklines=true,
breakatwhitespace=false,
title=\lstname,
escapeinside={},
keywordstyle=\color{deepblue},
emphstyle=\color{deepred},
stringstyle=\color{deepgreen}
%morekeywords={models, lambda, forms}
}

% Presentation configuration
\mode<presentation>{\usetheme{Madrid}}
\definecolor{tealblue}{rgb}{0, 0.5, 0.5}
\usecolortheme[named=tealblue]{structure}
\useinnertheme{circles} % circles, rectanges, rounded, inmargin
\usefonttheme[onlymath]{serif} % makes math fonts like the usual LaTeX ones
\setbeamercovered{transparent=4} % transparent
\setbeamertemplate{caption}{\raggedright\insertcaption\par} % Remove the word "Figure" from caption %\setbeamertemplate{caption}[default]
\setbeamertemplate{navigation symbols}{} % don't put navigation tools at the bottom (alternatively \beamertemplatenavigationsymbolsempty)
\graphicspath{ {../images/} }

% Titlepage
\title[Python for scientific research]{Python for scientific research}
\subtitle{Introduction}
\author{John Joseph Valletta}
\date[June 2017]{June 2017}
\institute[]{University of Exeter, Penryn Campus, UK}
\titlegraphic{
\hfill
\includegraphics[width=\textwidth, keepaspectratio]{logo.jpg}}

%=============================================================================%
%=============================================================================%
% Start of Document
%=============================================================================%
%=============================================================================%
\begin{document}

%=============================================================================%
%=============================================================================%
\begin{frame}
\titlepage
\end{frame}

%=============================================================================%
%=============================================================================%
\begin{frame}{Acknowledgements}
\begin{itemize}\addtolength{\itemsep}{\baselineskip}
	\item The workshop is funded by Exeter's researcher-led initiative award 
	\item Thanks to \href{http://www.exeter.ac.uk/biomedicalhub/team/drjeremymetz/}{Jeremy Metz} for sharing his \href{https://metzjp.bitbucket.io/}{notes} used in the Biomedical Informatics Hub, from which I borrowed some examples 
	\item Last but not least, big thanks to Mario Recker, Thomas Holding, Warren Tennant and James Clewett for helping out putting this workshop together
\end{itemize}
\vfill
\includegraphics[width=\textwidth, keepaspectratio]{logo.jpg}

\end{frame}

%=============================================================================%
%=============================================================================%
\begin{frame}{Housekeeping}

\centering
\includegraphics<1>[width=\textwidth]{day1.pdf}
\includegraphics<2>[width=\textwidth]{day2.pdf}

\end{frame}

%=============================================================================%
%=============================================================================%
\begin{frame}{References}

\begin{itemize}\addtolength{\itemsep}{\baselineskip}
	\item \href{https://python.swaroopch.com/}{A Byte of Python}
	\item \href{http://greenteapress.com/wp/think-python/}{Think Python}
	\item \href{http://www.southampton.ac.uk/~fangohr/training/python/pdfs/Python-for-Computational-Science-and-Engineering.pdf}{Python for Computational Science and Engineering}
	\item \href{https://hplgit.github.io/primer.html/doc/pub/half/book.pdf}{A Primer on Scientific Programming with Python}
	\item \href{https://www.kevinsheppard.com/Python_for_Econometrics}{Introduction to Python for Econometrics, Statistics and Numerical Analysis}
\end{itemize}

\end{frame}

%=============================================================================%
%=============================================================================%
\begin{frame}{What is Python?}
% Monty Python figure
\centering
\includegraphics[width=0.35\textwidth]{monty_python.jpg}

\small
\begin{itemize}\addtolength{\itemsep}{.2\baselineskip}
	\item<2-> A scripted high-level programming language created by \href{https://en.wikipedia.org/wiki/Guido_van_Rossum}{Guido Van Rossum} and named after \href{https://en.wikipedia.org/wiki/Monty_Python's_Flying_Circus}{Monty Python's Flying Circus}
	\item<3-> Easy-to-use, versatile and with an emphasise on readability
	\item<4-> It has a minimalistic English-like syntax, relying on indentation instead of curly brackets, semicolons etc. 
\end{itemize}
\normalsize

\end{frame}

%=============================================================================%
%=============================================================================%
\begin{frame}{Why Python?}
The \href{http://www.tiobe.com/tiobe-index/}{TIOBE index} is a measure of the popularity of programming languages

% Tiobe index
\centering
\includegraphics[width=.85\textwidth]{tiobe.png}
\end{frame}

%=============================================================================%
%=============================================================================%
\begin{frame}{Why Python?}
%. Here are some reasons for its popularity:

\begin{itemize}\addtolength{\itemsep}{0.5\baselineskip}
	\item<1-> It is free! No licence costs
	\item<2-> Runs on all platforms (Mac, Windows, Linux)
	\item<3-> Because of it's ease of programming (e.g no neeed to worry about memory allocation), Python minimises development effort
	\item<4-> A huge number of \href{https://pypi.python.org/pypi}{libraries}, written by an active \href{https://www.python.org/community/}{community}  
	\item<5-> Python can ``glue" together functions written in C/C++ and Fortran to speed things up (we can also call R and MATLAB functions)
	\item<6-> Compared to other high-level scientific languages such as MATLAB and R, Python offers a much wider range of additional functionality (e.g \href{https://www.djangoproject.com/}{web} and \href{https://wiki.python.org/moin/TkInter}{GUI} development) %hence the nickname ``the swiss army knife" of programming languages. 
\end{itemize}

\end{frame}

%=============================================================================%
%=============================================================================%
\begin{frame}{Horses for courses}

\begin{itemize}\addtolength{\itemsep}{0.5\baselineskip}
	\item<1-> Python is becoming the de facto standard for exploratory and interactive scientific research\\
	\item[]<1-> \textbf{BUT}
	\item<2-> Python is no programming silver bullet
	\item<3-> Your application will ultimately dictate the tool (and a mixture of more than one language \emph{is} ok). For example:\\
	\begin{itemize}\addtolength{\itemsep}{0.8\baselineskip}
		\item<4-> MATLAB excels at interfacing with hardware, e.g generating \href{https://uk.mathworks.com/products/hdl-coder.html}{hardware description language (HDL) code} to configure an integrated circuit board or connecting to a \href{https://uk.mathworks.com/products/daq.html}{data acquisition card}
		\item<5-> R is great for data wrangling and visualisation, and statistical modelling
		\item<6-> \href{http://mc-stan.org/}{Stan} (a probabilistic programming language) is an excellent choice for performing full Bayesian statistical inference
	\end{itemize}
\end{itemize}

\end{frame}

%=============================================================================%
%=============================================================================%
\begin{frame}{Why do \textit{you} want to learn Python?}
% Word cloud
\centering
\includegraphics[width=\textwidth]{wordcloud.png}
\end{frame}

%=============================================================================%
%=============================================================================%
\begin{frame}[fragile]
\frametitle{Executing Python code: No frills Python interpreter}

\begin{itemize}\addtolength{\itemsep}{0.5\baselineskip}
	\item Type \texttt{python} in your terminal window to invoke the interpreter
	\item Any Python code you type in is executed once you press enter  
\end{itemize}

\centering
\includegraphics[width=0.8\textwidth]{python_interpreter.png}

\begin{itemize}\addtolength{\itemsep}{0.5\baselineskip}
	\item Alternatively if your code is written in a text file, e.g \texttt{my\_script.py}:
\end{itemize}

\begin{lstlisting}[style=bash]
python my_script.py
\end{lstlisting}

\end{frame}

%=============================================================================%
%=============================================================================%
\begin{frame}[fragile]
\frametitle{Executing Python code: IPython interpreter}
\begin{itemize}
	\item IPython is an interactive shell (similar to R Console), adding ``frills" to the vanilla interpreter, such as:

	\begin{itemize}
		\item syntax highlighting (making it easier to read code)
		\item tab auto-completion (minimises typeos and lists available functions)  
	\end{itemize}

	\centering
	\includegraphics[width=0.6\textwidth]{ipython.png}

	%\item The IPyton interactive shell is what powers Jupyter in the background (i.e code that I write in these notes is interpreted and executed by the IPython shell)
\end{itemize}
\end{frame}

%=============================================================================%
%=============================================================================%
\begin{frame}[fragile]
\frametitle{Executing Python code: Spyder IDE}
\begin{itemize}
	\item Spyder is an integrated development environment (IDE) for scientific computing, akin to \href{https://www.rstudio.com/}{RStudio} and \href{https://uk.mathworks.com/products/matlab.html}{MATLAB} 
	\item One place to write, execute and debug code, and explore variables

	\centering
	\includegraphics[width=0.85\textwidth]{spyder.png}
\end{itemize}
\end{frame}

%=============================================================================%
%=============================================================================%
\begin{frame}[fragile]
\frametitle{The Zen of Python}
\begin{itemize}
	\item Coding standards are important in \textit{every} programming language%, but particular emphasis is placed in Python
	\item \href{https://www.python.org/dev/peps/pep-0008/}{PEP 8} is a style guide for python code  
	%\item If in doubt, be \textbf{consistent} in the way you structure your code
\end{itemize}

\scriptsize
\begin{verbatim}
Beautiful is better than ugly.
Explicit is better than implicit.
Simple is better than complex.
Complex is better than complicated.
Flat is better than nested.
Sparse is better than dense.
Readability counts.
Special cases aren't special enough to break the rules.
Although practicality beats purity.
Errors should never pass silently.
Unless explicitly silenced.
In the face of ambiguity, refuse the temptation to guess.
There should be one-- and preferably only one --obvious way to do it.
Although that way may not be obvious at first unless you're Dutch.
Now is better than never.
Although never is often better than *right* now.
If the implementation is hard to explain, it's a bad idea.
If the implementation is easy to explain, it may be a good idea.
Namespaces are one honking great idea -- let's do more of those!
\end{verbatim}
\normalsize
\end{frame}

%=============================================================================%
%=============================================================================%
\begin{frame}{Python 2.x vs 3.x}
\centering
\includegraphics[width=.6\textwidth]{python2vs3.pdf}

\begin{itemize}\addtolength{\itemsep}{0.3\baselineskip}
	\item<1-> Python 2.x and Python 3.x are the two main versions of Python
	\item<2-> \href{https://wiki.python.org/moin/Python2orPython3}{Python 2.x is legacy, Python 3.x is the present and future of the language}
	\item<3-> However, not all Python 3.x code is backwards-compatible
	\item<4-> Be aware of \href{http://sebastianraschka.com/Articles/2014_python_2_3_key_diff.html}{key differences} between the two
	\item<5-> Here we will use Python 3.x, the language actively being developed
\end{itemize}
\end{frame}

%=============================================================================%
%=============================================================================%
\begin{frame}{Installing Python}
\begin{itemize}\addtolength{\itemsep}{0.5\baselineskip}
	\item The easiest way to get started is to download and install a cross-platform Python distribution such as:
	\begin{itemize}
		\item \href{https://www.continuum.io/downloads}{Anaconda}
		\item \href{https://store.enthought.com/downloads/}{Enthought Canopy}
	\end{itemize}
	\item These distributions contain most libraries you need to get started
	\item Here we will use Anaconda which should be installed on your machines
\end{itemize}

\begin{figure}
\includegraphics[width=\textwidth]{anaconda.png}
\end{figure}

\end{frame}

%=============================================================================%
%=============================================================================%
% End of Document
%=============================================================================%
%=============================================================================%
\end{document}

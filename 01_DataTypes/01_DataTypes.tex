%=============================================================================%
% Author: 	John Joseph Valletta
% Date: 	14/03/2017
% Title: 	Python workshop: Data types
%=============================================================================%

%=============================================================================%
% Preamble
%=============================================================================%
% Libraries
\documentclass[pdf]{beamer}
\usepackage[export]{adjustbox}
\usepackage{framed}
\usepackage{color}
\definecolor{dkgreen}{rgb}{0,0.6,0}
\definecolor{gray}{rgb}{0.5,0.5,0.5}
\definecolor{mauve}{rgb}{0.58,0,0.82}
\definecolor{deepblue}{rgb}{0,0,0.5}
\definecolor{deepred}{rgb}{0.6,0,0}
\definecolor{deepgreen}{rgb}{0,0.5,0}
\definecolor{lightgray}{rgb}{0.92,0.92,0.92}
\usepackage{listings} % to insert code
\usepackage{textpos} % textblock
\usepackage{hyperref}
\hypersetup{colorlinks=true, urlcolor=blue, linkcolor=black} 

% Listing set up
% Python
\lstdefinestyle{python}{
language=python,
formfeed=\newpage,
basicstyle=\scriptsize\ttfamily,
commentstyle=\color{deepgreen},%\color{gray},
%numbers=left,
%numberstyle=\tiny\color{gray},
stepnumber=1,
numbersep=5pt,
backgroundcolor=\color{lightgray},%\color{white},
showspaces=false,
showstringspaces=false,
showtabs=false,
frame=lines,
tabsize=4,
captionpos=b,
breaklines=true,
breakatwhitespace=false,
title=\lstname,
escapeinside={},
keywordstyle=\color{deepblue},
emphstyle=\color{deepred},
stringstyle=\color{mauve},
morekeywords={as, lambda, in, True, False}
}

% Presentation configuration
\mode<presentation>{\usetheme{Madrid}}
\definecolor{tealblue}{rgb}{0, 0.5, 0.5}
\usecolortheme[named=tealblue]{structure}
\useinnertheme{circles} % circles, rectanges, rounded, inmargin
\usefonttheme[onlymath]{serif} % makes math fonts like the usual LaTeX ones
\setbeamercovered{transparent=4} % transparent
\setbeamertemplate{caption}{\raggedright\insertcaption\par} % Remove the word "Figure" from caption %\setbeamertemplate{caption}[default]
\setbeamertemplate{navigation symbols}{} % don't put navigation tools at the bottom (alternatively \beamertemplatenavigationsymbolsempty)
\graphicspath{ {../images/} }

% Titlepage
\title[Python for scientific research]{Python for scientific research}
\subtitle{Built-in data types}
\author{John Joseph Valletta}
\date[June 2017]{June 2017}
\institute[]{University of Exeter, Penryn Campus, UK}
\titlegraphic{
\hfill
\includegraphics[width=\textwidth, keepaspectratio]{logo.jpg}}

%=============================================================================%
%=============================================================================%
% Start of Document
%=============================================================================%
%=============================================================================%
\begin{document}

%=============================================================================%
%=============================================================================%
\begin{frame}
\titlepage
\end{frame}

%=============================================================================%
%=============================================================================%
\begin{frame}{Declaring variables}

\begin{itemize}\addtolength{\itemsep}{0.5\baselineskip}
	\item<1-> Variables that contain data (numbers, characters etc.) are the fundamental building blocks of any language
	\item<2-> Variables in Python are dynamically typed; called duck typing\\[1em]
	\textit{If it walks like a duck and talks like a duck, then it's a duck!}
	\item<3-> You can name your variables pretty much anything you want but:\\
	\begin{itemize}\addtolength{\itemsep}{0.5\baselineskip}
		\item<4-> Be \textbf{consistent} in your naming convention (see \href{https://www.python.org/dev/peps/pep-0008/}{PEP 8})
		\item<5-> Python is \textbf{case sensitive}; \texttt{genename} and \texttt{geneName} are different variables
		\item<6-> You cannot use Python \textbf{reserved words/keywords}
		\item<7-> If you're an R user, \textbf{DO NOT} use `.' in your variable names i.e \texttt{gene.name} is not a valid variable name 
	\end{itemize} 
\end{itemize}

\end{frame}
%=============================================================================%
%=============================================================================%
\begin{frame}[fragile]
\frametitle{Core data types}

\begin{itemize}\addtolength{\itemsep}{0.5\baselineskip}

\item<1-> \textbf{Integers: \texttt{int}}\\
\begin{lstlisting}[style=python]
NGenes = 1500 # number of genes measured
type(NGenes) # int
\end{lstlisting}
	
\item<2-> \textbf{Floating point: \texttt{float}}
\begin{lstlisting}[style=python]
Km = 0.015 # Michaelis constant for chymotrypsin
type(Km) # float
\end{lstlisting}
	
\item<3-> \textbf{Complex numbers: \texttt{complex}}
\begin{lstlisting}[style=python]
power = 10 + 2j # 10=active/real power; 2=reactive power
type(power) # complex
\end{lstlisting}

\end{itemize}

\end{frame}

%=============================================================================%
%=============================================================================%
\begin{frame}[fragile]
\frametitle{Core data types}

\begin{itemize}\addtolength{\itemsep}{0.5\baselineskip}

\item<1-> \textbf{Boolean: \texttt{bool}}
\begin{lstlisting}[style=python]
isTransFactor = True # is protein a transcription factor?
type(isTransFactor) # bool
\end{lstlisting}

\item<2-> \textbf{Strings: \texttt{str}}
\begin{lstlisting}[style=python]
motif = "AATCAGTT" # DNA sequence motif
type(motif) # str
\end{lstlisting}

\end{itemize}

\end{frame}

%=============================================================================%
%=============================================================================%
\begin{frame}[fragile]
\frametitle{Container data types}

\begin{itemize}\addtolength{\itemsep}{-0.5\baselineskip}
\item<1-> \textbf{Lists: \texttt{list}}
\begin{lstlisting}[style=python]
# Interesting genes
geneNames = ["Irf1", "Ccl3", "Il12rb1", "Ifng", "Cxcl10"]
type(geneNames) # list
\end{lstlisting}

\item<2-> \textbf{Tuples: \texttt{tuple}}
\begin{lstlisting}[style=python]
# Interesting genes
geneNames = ("Irf1", "Ccl3", "Il12rb1", "Ifng", "Cxcl10")
type(geneNames) # tuple
\end{lstlisting}

\item<3-> \textbf{Dictionary: \texttt{dict}}
\begin{lstlisting}[style=python]
# My favourite gene
favGene = {"Symbol": "Ifng", "Name": "Interferon", "ID": 3458}
type(favGene) # dict
\end{lstlisting}

\end{itemize}
\end{frame}

%=============================================================================%
%=============================================================================%
\begin{frame}[fragile]
\frametitle{Container data types}

\begin{itemize}\addtolength{\itemsep}{0.5\baselineskip}

\item<1-> \textbf{Sets: \texttt{set, frozenset}}
\begin{lstlisting}[style=python]
# Sets are mutable
languages = set(["Python", "R", "MATLAB", "C"])
type(languages) # set

# Frozensets are immutable
languages = frozenset(["Python", "R", "MATLAB", "C"])
type(languages) # frozenset
\end{lstlisting}

\item<2-> \textbf{Range: \texttt{range}}
\begin{lstlisting}[style=python]
# Create immutable sequence of numbers from 0 to 4
x = range(5)
type(x) # range
\end{lstlisting}

\end{itemize}

\end{frame}

%=============================================================================%
%=============================================================================%
\begin{frame}[fragile]
\frametitle{Mutable vs immutable objects}

\begin{itemize}

\item<1-> \textbf{Mutable} objects (\texttt{list, dict, set}) can be changed once assigned
\begin{lstlisting}[style=python]
# Lists are mutable
geneList = ["Irf1", "Ccl3", "Il12rb1"]
geneList[0]="Irf2" # change first gene to Irf2
\end{lstlisting}  

\item<2-> \textbf{Immutable} objects \textit{cannot} be changed once assigned 
\begin{lstlisting}[style=python]
# Tuples are immutable
geneTuple = ("Irf1", "Ccl3", "Il12rb1")
geneTuple[0]="Irf2"
TypeError: 'tuple' object does not support item assignment
\end{lstlisting}  

\item<3-> However you can replace an \textbf{immutable} object with a new one 
\begin{lstlisting}[style=python]
geneTuple = ("Irf1", "Ccl3", "Il12rb1")
# Replace with a new object
geneTuple = ("Irf2", "Ccl3", "Il12rb1")
\end{lstlisting} 

\end{itemize}

\end{frame}

%=============================================================================%
%=============================================================================%
\begin{frame}[fragile]
\frametitle{Methods of objects}

\begin{itemize}\addtolength{\itemsep}{0.5\baselineskip}
	\item A Python variable is called an \textbf{object}
	\item Every object has \textbf{methods} (functions) associated with it
	\item These methods are called using the dot notation (\texttt{`.'})
\end{itemize}
\begin{lstlisting}[style=python]
# DNA sequence motif
motif = "AATCAGTT"

# Use the count method to count occurrence of nucleotide "T" 
motif.count("T") # 3

# Use the lower method to convert to lower case
motif.lower() # "aatcagtt"
\end{lstlisting} 

\end{frame}

%=============================================================================%
%=============================================================================%
% End of Document
%=============================================================================%
%=============================================================================%
\end{document}
